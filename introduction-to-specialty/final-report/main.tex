\documentclass[referat, times]{SCWorks}

\usepackage{preamble}

\begin{document}

% Кафедра (в родительном падеже)
\chair{информатики и программирования}

% Тема работы
\title{ПРОДУКТОВАЯ АНАЛИТИКА}

% Курс
\course{1}

% Группа
\group{151}

% Специальность/направление код - наименование
\napravlenie{09.03.04 Программная инженерия}

\studenttitle{студентки}
% Фамилия, имя, отчество в родительном падеже
\author{Ореховой Алины Сергеевны}

% Год выполнения отчета
\date{2023}

\maketitle

\tableofcontents

\intro
Для улучшения качества пользовательского опыта компании используют продуктовую аналитику. Процессы работы с данными автоматизируются и отслеживать действия пользователей становится проще. Команда принимает решения на основе полученной информации, тем самым максимизируя качество продукта и прибыль компании.

Для IT"=компаний особенно важно понимать, как используется созданный продукт, и именно это позволяет воплотить продуктовая аналитика. Работа продуктового аналитика заключается в сборе и обработке этих данных, выделении главной информации и ее визуализации.

Продуктовая аналитика показывает компаниям, что на самом деле делают их пользователи, а не только то, что они говорят. Здесь мы говорим о проявляемом поведении (revealed behavior), и оно крайне красноречиво само по себе. Наличие аналитики позволяет командам разработчиков копать глубже, чем опросы и интервью с пользователями, на которые влияет человеческий фактор. 

По данным McKinsey, <<Компании, которые используют продуктовую аналитику на полную, почти в два раза чаще говорят, что опережают своих конкурентов с точки зрения прибыли, чем компании, которые ее не используют>>\cite{mckinsey}. 

Продуктовую аналитику иногда относят к бизнес-аналитике, но это не совсем верно. Бизнес-аналитика говорит о делах в бизнесе на сегодняшний момент: сколько продали в прошлом месяце, на этой неделе, сколько потратили на маркетинг и был ли достигнут эффект. А продуктовая аналитика даёт понимание текущего положения дел в конкретном продукте. Её использование эффективно тогда, когда её используют в режиме реального времени: оценивают имеющиеся данные, строят продуктовые гипотезы, проверяют их и оценивают эффективность продукта здесь и сейчас\cite{businessAnalysis}.

\section {Задачи product-аналитика в компании}
Платформы с инструментами для работы продуктового аналитика обычно выполняют две основные функции:
\begin{enumerate}
    \item Данные отслеживания: запись посещений, событий, действий.
    \item Анализ данных: визуализация данных с помощью дашбордов и отчетов.
\end{enumerate}

Аналитик собирает отслеженные данные и представляет их в наглядной форме. Полученные графики помогают компании отвечать на такие вопросы, как:

\begin{itemize}
    \item Какова пользовательская демография?
    \item Что пользователи обычно делают на сайти?
    \item Как можно сократить отток пользователей?
\end{itemize}

Полученные из статистики выводы используются в принятии маркетинговых и продуктовых решений. Таким образом, общие цели продуктовой аналитики:
\begin{itemize}
    \item Повысить метрику retention;
    \item Выделить самых прибыльных пользователей;
    \item Оптимизировать маркетинговый бюджет;
    \item Понять, как имено используется сайт или приложение;
    \item Определить проблемы пользователей;
    \item Уменьшить отток пользователей.
\end{itemize}

После этого команда строит гипотезы и пытается доказать их или опровергнуть. Продуктовая гипотеза может выглядеть как утверждение в форме: <<Поможет ли всплывающее окно увеличить количество подписчиков на 25\%?>>
Через некоторое время у команды накапливается целое хранилище проверенных данных, которые снова смогут использоваться в исследованиях. Чем большее количество таких итераций переживет продукт "--- тем более удобным он будет в итоге\cite{tasksInCompany}.

\section{Навыки, необходимые product-аналитику}

\subsection{SQL}
SQL (Structured Query Language) "--- язык, отвечающий за запросы к базам данных. С помощью SQL аналитик выгружает и предобрабатывает данные. Для этого необходимо:
\begin{enumerate}
    \item Знать и понимать как работает фильтрация, чтобы выгружать лишь те данные, которые необходимы в данном случае;
    \item Понять как работают функции агрегирования "--- что очень напоминает работу с Excel;
    \item SQL Join "--- один из наиболее вероятных вопросов в собеседовании на позицию продуктового аналитика. Join отвечает за то, как можно соединить 2 таблицы, как они взаимосвязаны;
    \item Подзапросы "--- в какой-то мере это можно понимать как динамическую фильтрацию, работающую в режиме реального времени;
    \item Понимать оконные функции "--- несколько более сложная тема, так же часто встречающаяся в собеседованиях (скорее даже на позицию Middle)\cite{sql}.
\end{enumerate}

\subsection{Python}
Используется для анализа предобработанных до этого данных. 
Для работы аналитика достаточно следующих библиотек:

\begin{itemize}
    \item SciPy "--- для различных вычислений;
    \item NumPy "--- для работы с числами и формулами;
    \item Pandas "--- для работы с таблицами;
    \item Matplotlib "--- для построения графиков.
\end{itemize}

\subsection{Статистика, А/Б тестирование}
A/B"=тестирование "--- это широко применяемый в аналитике
статистический метод, который сравнивает вовлеченность или реакцию как
минимум двух вариантов чего-либо (например, веб"=страницы), чтобы
определить, обеспечивает ли одна версия более эффективный или заранее
заданный результат (например, связь с исследовательской группой), чем
другие. A/B"=тестирование можно использовать как для проверки
производительности незначительных эстетических различий, таких как цвет
кнопки веб-сайта (синий или красный), или более существенных изменений,
таких как две разные фотографии на главной странице веб"=сайта. Важно
отметить, что несколько A/B"=тестов могут быть развернуты параллельно или
последовательно, что позволяет итеративно, быстро менять и улучшать вебсайт или медиа"=контент на основе эмпирических данных\cite{miller2021b}.

A/B"=тестирование необходимо для получения объективной информации
о всевозможных способах улучшения текущей версии продукта. Реальные
данные позволяют компании быстрее продвигаться на рынке. Только сами
пользователи лучше всех аналитиков знают, что им нужно здесь и сейчас\cite{ABtest}.

\subsection{Средства визуализации данных}
Любая продуктовая аналитика не <<живет>> без дашбордов. Дашборд "--- это информационная панель, которая получает данные из других систем и отображает их в понятном виде. На дашбордах используют текст, графики, диаграммы и другие средства визуализации. Панели получают данные и обновляются автоматически с заданным интервалом или даже в режиме реального времени. Создание такой панели "--- прямая задача продуктового аналитика.

\subsection{Продуктовое понимание}
Продуктовое понимание включает в себя метрики, продуктовые гипотезы, методологии разработки и их взаимосвязь.
Для улучшения продуктового понимания можно читать статьи в блогах, смотреть на YouTube видео работающих аналитиков, или, например, лекции Яндекса по управлению проектами и продуктами (их легко найти).

\section{Карьерные перспективы product-аналитика}
Традиционно аналитики востребованы в ИТ- и digital-сферах, но в последние годы спрос на них активно растёт и в других отраслях бизнеса: от ритейла до производства.

Плохая новость для начинающих погружение в область людей заключается в том, что набор компетенций, который требуют от аналитиков, постоянно расширяется.

Хорошая же новость (по крайней мере, для аналитиков) заключается в том, что спрос на профессионалов значительно превышает предложение. Однако, как и в других сферах, это распространяется действительно только на профессионалов "--- пробиться среди сотен неопытных продуктовых аналитиков всех возрастов остается трудно (что в целом характерно для IT"=рынка сейчас).

Согласно исследованию компании «Нормальные исследования», продуктовые аналитики, проработавшие до 1 года зарабатывают в среднем 83 000 рублей, от 1 до 2 лет "--- 134 000, от 2 до 3 "--- 236 000, от 3 до 6 лет "--- 274 000, и наконец более 6 лет "--- 300 000. Исследователи уточняют, что заработную плату рассчитывали по медиане, чтобы избежать искажения данных, связанных со значительными выбросами (пиковые значения). Например, небольшая часть респондентов зарабатывает от 400 тысяч до 1 млн рублей, поэтому при использовании этих данных среднее значение было бы искажено\cite{research}.

Говоря о том, куда дальше планируют развиваться аналитики, 37\% респондетов планируют стать экспертами в своей области, 24\% хотят стать менеджерами-управленцами, 18\% желает начать свой бизнес, 12\% думают сменить специализацию аналитики, а 5\% "--- сменить профессию полностью. Лишь 4\% не планируют никаких изменений и счастливы с тем, что имеют сейчас.

\section{Примеры успешной реализации product-аналитики в компаниях}
Рассмотрим работу команды аналитиков в реальных компаниях.
\subsection{Яндекс.Погода}
В Яндексе есть сервис погоды. Команда решила попробовать показывать уведомление с текущей погодой пользователю на экране заставки. Как измерить эффективности фичи и почему retention в данном случае плохая метрика?

Цель уведомления "--- принести пользу потребителям и, как следствие, увеличить лояльность к компании. При этом ожидалось, что метрика retention вырастет вследствие осознанного перехода пользователя, а не из-за случайного или импульсивного нажатия на уведомление.

Однако не было учтена специфика продукта "--- на уведомление с погодой не обязательно нажимать, ведь все данные видны сразу. Тогда решили ввести уточненный retention: стали считать пользователей, увидевших сообщение о погоде, но:

\begin{enumerate}
    \item Не нажимавших на него вообще;
    \item Нажавших на уведомление, но не ограничившихся просмотром погоды, а продолживших делать другие дела в браузере 
    (то есть пользователь и так собирался поработать в браузере, уведомление лишь ускорило начало сессии);
\end{enumerate}

Если такой retention растет, значит уведомление приносит пользу и <<растит>> лояльность. Однако, как можно заметить, использование retention как целевой метрики сопряжено своими трудностями, поэтому в данном случае лучше сразу смотреть на более общие метрики, такие как суммарные переходы на сайты\cite{yandexWeather}.

\subsection{Airbnb}
Airbnb — онлайн-площадка для размещения и поиска краткосрочной аренды частного жилья по всему миру (65 000 городов в 191 стране). Огромная часть успеха компании "--- результат налаженной работы команды специалистов по обработке данных.

Начало обработки данных продукта лежит непосредственно в компетенции продуктовых аналитиков, и только потом уже попадает в руки Дата"=саентистов.
Работа по анализу продукта носит исследовательский характер. Говоря о <<продукте>>, имеется в виду главным образом веб"=сайт Airbnb и мобильное приложение. Работа носит в некоторой степени является творческой "--- не совсем понятно, что нужно найти, известно лишь конечная цель состоит в том, чтобы найти возможности сделать продукт лучше. Такие вопросы, как <<Какие категории гостей останавливаются на Airbnb в этих регионах?>>, <<Почему некоторые новые хостинги не бронируются?>>, <<В каких городах предложение ограничено?>> "--- все это примеры вопросов, с которыми сталкивается продуктовый аналитик в Airbnb. Выводы, почерпнутые из этой работы, часто непосредственно приводят к появлению идей о новых продуктах и гипотез о поведении пользователей, которые попадают на следующий этап обработки данных "--- эксперементирование. 

Экспериментирование (также называемое A/B"=тестированием) играет важную роль в разработке продукта, основанного на данных. Цель экспериментов - подтвердить или опровергнуть гипотезы, которые есть у команды, чтобы улучшить пользовательский опыт. Если гипотезы подтвердятся, изменения внесут в приложение/сайт в режиме реального времени. Почти все гипотезы и идеи в Airbnb проверяются с помощью контролируемых экспериментов, где задания распределяются случайным образом. Прежде чем представить миру новые функции продукта,  проверяется, влияют ли новые дизайны или информационные продукты на ключевые показатели, такие как количество бронирований, заявки на обслуживание клиентов, оценки отзывов, отток посетителей и десятки других показателей. Кроме того, эксперименты "--- это способ избавить команду от лишней работы в случае ошибки, ведь сам процесс разработки требует затрат большого количества ресурсов\cite{airbnb}. 

\section{Основные проблемы при внедрении product-аналитики в компанию}
\subsection{Новизна профессии}
В связи с новизной профессии продуктового аналитика у компаний еще не успели сформироваться стандарты по процессам для работы данных специалистов. Это приводит к потере эффективности аналитиков, неудобству заказчиков во взаимодействии с ними, нарастанию стрессовых состояний и потере мотивации.

\subsection{Понимание контекста}
Понимание контекста "--- один из самых важных аспектов работы аналитика. Без него сложно хорошо сделать задачу, не впасть в фрустрацию и в целом оценить полезность проделанной работы. Часто заказчики пытаются помогать аналитикам, когда прописывают, что именно аналитику надо сделать, лишая аналитика инициативы придумать решение лучше, отбирают у него творческую составляющую работы, делая из него просто исполнителя. Как следствие, это тормозит погружение продуктового аналитика в контекст бизнеса. Ведь он не понимает, для чего он смотрит динамику конверсий, выгружает данные и за чем на самом деле хочет следить продуктовый менеджер, когда просит его автоматизировать уже придуманную формулу на дашборд.

\subsection{Проблемы коммуникации}
То, как аналитик оформляет результат своей работы, "--- еще один важный аспект. Часто ему сложно переключиться с языка сложных терминов на язык, понятный заказчику, оформить все в доступном виде. Поскольку работа аналитика состоит в доставке не основной ценности бизнесу, а дополнительной, то зачастую заказчикам проще принять решение самостоятельно, чем добиваться от аналитика простого разъяснения.

\subsection{Трудности с менеджментом}
Аналитики, оказавшиеся без регулярного менеджмента, к которым заказчики обращаются напрямую, могут также столкнуться с большим потоком задач с горящими сроками. Оценить важность <<срочной>> задачи, когда давят стресс и дедлайн, бывает сложно. А отказать в задаче для многих еще сложнее\cite{pilyavskaya}.



\conclusion
Продуктовая аналитика является неотъемлемой частью успешной стратегии развития продукта. Она позволяет собирать и анализировать данные о поведении пользователей, что помогает команде разработки продукта принимать обоснованные решения и улучшать продукт в соответствии с потребностями клиентов.

Кроме того, продуктовая аналитика помогает определить, какие маркетинговые кампании наиболее эффективны, а какие необходимо изменить или дополнить. Это позволяет улучшить конверсию и увеличить количество пользователей продукта.

Однако необходимо учитывать и проблемы, с которыми может столкнуться аналитик и вся команда "--- отстуствие выработанных стандартов по процессам работы аналитиков, понимание ими контекста продуктов, проблемы коммуникации между разработчиками и бизнесом, и даже банальные трудности с менеджментом.

Но несмотря на все препятствия "--- наличие продуктового аналитика в команде обычно окупает свои немаленькие затраты, ведь этот человек ответственен за успешность релиза продукта в каком"=то смысле больше всех остальных. 

Сама по себе профессия довольно интересная, потому что охватывает широкий спектр задач и предполагает соответствующую ответственность. При этом продуктовая аналитика является настоящим полем для творческих экспериментов (особенно при проведении А/Б"=тестирований).  

% Отобразить все источники. Даже те, на которые нет ссылок.
\nocite{*}

\inputencoding{cp1251}
\bibliographystyle{gost780uv}
\bibliography{thesis}
\inputencoding{utf8}

% Окончание основного документа и начало приложений
% Каждая последующая секция документа будет являться приложением
\appendix

\end{document}