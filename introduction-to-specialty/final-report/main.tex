\documentclass[referat, times]{SCWorks}

\usepackage{preamble}

\begin{document}

% Кафедра (в родительном падеже)
\chair{информатики и программирования}

% Тема работы
\title{ПРОДУКТОВАЯ АНАЛИТИКА}

% Курс
\course{1}

% Группа
\group{151}

% Специальность/направление код - наименование
\napravlenie{09.03.04 Программная инженерия}

\studenttitle{студентки}
% Фамилия, имя, отчество в родительном падеже
\author{Ореховой Алины Сергеевны}

% Год выполнения отчета
\date{2023}

\maketitle

\tableofcontents

\intro
вот
\section {Задачи product-аналитика в компании}
% здесь \section, \subsection ...
чтоо

\section{Методы product-аналитики}
ммм

\section{Навыки, необходимые product-аналитику}

\subsection{SQL}
чтоо

\subsection{Python}
как 

\subsection{Статистика}
вот так

\subsection{Средства визуализации данных}
ммм

\subsection{Продуктовое понимание}
да

\section{Карьерные перспективы product-аналитика}
какие

\section{Примеры успешной реализации product-аналитики в компаниях}
которые

\section{Основные проблемы при внедрении product-аналитики в компанию}
какие



\conclusion
ммм
% Отобразить все источники. Даже те, на которые нет ссылок.
\nocite{*}

\inputencoding{cp1251}
\bibliographystyle{gost780uv}
\bibliography{thesis}
\inputencoding{utf8}

% Окончание основного документа и начало приложений
% Каждая последующая секция документа будет являться приложением
\appendix

\end{document}