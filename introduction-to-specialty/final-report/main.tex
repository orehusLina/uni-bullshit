\documentclass[referat, times]{SCWorks}

\usepackage{preamble}

\begin{document}

% Кафедра (в родительном падеже)
\chair{информатики и программирования}

% Тема работы
\title{ПРОДУКТОВАЯ АНАЛИТИКА}

% Курс
\course{1}

% Группа
\group{151}

% Специальность/направление код - наименование
\napravlenie{09.03.04 Программная инженерия}

\studenttitle{студентки}
% Фамилия, имя, отчество в родительном падеже
\author{Ореховой Алины Сергеевны}

% Год выполнения отчета
\date{2023}

\maketitle

\tableofcontents

\intro
вот
\section {Задачи product-аналитика в компании}
% здесь \section, \subsection ...
чтоо

\section{Методы product-аналитики}
ммм

\section{Навыки, необходимые product-аналитику}

\subsection{SQL}
чтоо

\subsection{Python}
как 

\subsection{Статистика}
вот так

\subsection{Средства визуализации данных}
ммм

\subsection{Продуктовое понимание}
да

\section{Карьерные перспективы product-аналитика}
какие

\section{Примеры успешной реализации product-аналитики в компаниях}
Рассмотрим несколько реальных кейсов работы продуктового аналитика в компании.
\subsection{Яндекс.Погода}
В Яндексе есть сервис погоды. Команда решила попробовать показывать уведомление с текущей погодой пользователю на экране заставки. Как измерить эффективности фичи и почему retention в данном случае плохая метрика?

Цель уведомления "--- принести пользу потребителям и, как следствие, увеличить лояльность к компании. При этом ожидалось, что метрика retention вырастет вследствие осознанного перехода пользователя, а не из-за случайного или импульсивного нажатия на уведомление.

Однако не было учтена специфика продукта "--- на уведомление с погодой не обязательно нажимать, ведь все данные видны сразу. Тогда решили ввести уточненный retention: стали считать пользователей, увидевших сообщение о погоде, но:

\begin{enumerate}
    \item Не нажимавших на него вообще;
    \item Нажавших на уведомление, но не ограничившихся просмотром погоды, а продолживших делать другие дела в браузере 
    (то есть пользователь и так собирался поработать в браузере, уведомление лишь ускорило начало сессии);
\end{enumerate}

Если такой retention растет, значит уведомление приносит пользу и <<растит>> лояльность. Однако, как можно заметить, использование retention как целевой метрики сопряжено своими трудностями, поэтому в данном случае лучше сразу смотреть на более общие метрики, такие как суммарные переходы на сайты\cite{yandexWeather}.

\section{Основные проблемы при внедрении product-аналитики в компанию}
\subsection{Новизна профессии}
В связи с новизной профессии продуктового аналитика у компаний еще не успели сформироваться стандарты по процессам для работы данных специалистов. Это приводит к потере эффективности аналитиков, неудобству заказчиков во взаимодействии с ними, нарастанию стрессовых состояний и потере мотивации.

\subsection{Понимание контекста}
Понимание контекста "--- один из самых важных аспектов работы аналитика. Без него сложно хорошо сделать задачу, не впасть в фрустрацию и в целом оценить полезность проделанной работы. Часто заказчики пытаются помогать аналитикам, когда прописывают, что именно аналитику надо сделать, лишая аналитика инициативы придумать решение лучше, отбирают у него творческую составляющую работы, делая из него просто исполнителя. Как следствие, это тормозит погружение продуктового аналитика в контекст бизнеса. Ведь он не понимает, для чего он смотрит динамику конверсий, выгружает данные и за чем на самом деле хочет следить продуктовый менеджер, когда просит его автоматизировать уже придуманную формулу на дашборд.

\subsection{Проблемы коммуникации}
То, как аналитик оформляет результат своей работы, "--- еще один важный аспект. Часто ему сложно переключиться с языка сложных терминов на язык, понятный заказчику, оформить все в доступном виде. Поскольку работа аналитика состоит в доставке не основной ценности бизнесу, а дополнительной, то зачастую заказчикам проще принять решение самостоятельно, чем добиваться от аналитика простого разъяснения.

\subsection{Трудности с менеджментом}
Аналитики, оказавшиеся без регулярного менеджмента, к которым заказчики обращаются напрямую, могут также столкнуться с большим потоком задач с горящими сроками. Оценить важность <<срочной>> задачи, когда давят стресс и дедлайн, бывает сложно. А отказать в задаче для многих еще сложнее\cite{pilyavskaya}.



\conclusion
ммм
% Отобразить все источники. Даже те, на которые нет ссылок.
\nocite{*}

\inputencoding{cp1251}
\bibliographystyle{gost780uv}
\bibliography{thesis}
\inputencoding{utf8}

% Окончание основного документа и начало приложений
% Каждая последующая секция документа будет являться приложением
\appendix

\end{document}