\begin{flushleft} \large\textbf{A} \end{flushleft}

\begin{enumerate}
    \setcounter{enumi}{0}

    \item \textbf{Agile} (\textit{Никита Барабанов}) "--- гибкий подход к разработке программного обеспечения, основанный на сотрудничестве и быстрой адаптации к изменениям.
    
    \item \textbf{AnimNotifier} (\textit{Леонид Сорокин}) "--- компонент в Unreal Engine, который используется для управления анимациями персонажей в играх. 
\end{enumerate}

\begin{flushleft} \large\textbf{B} \end{flushleft}

\begin{enumerate}
    \setcounter{enumi}{2}

    \item \textbf{Benefits} (с англ. <<льгота>>, <<выгода>>) (\textit{Никита Барабанов}) "--- преимущества или выгоды, которые получает компания или проект. 
    
    \item \textbf{Blueprints} (\textit{Леонид Сорокин}) "--- графический интерфейс для создания логики игры в Unreal Engine. 
\end{enumerate}

\begin{flushleft} \large\textbf{C} \end{flushleft}

\begin{enumerate}
    \setcounter{enumi}{4}

    \item \textbf{Chat GPT} (\textit{Алексей Кузьмин}) "--- 
    модель генерации текста, использующая метод глубокого обучения и нейронные сети для создания текстовых сообщений, которые могут быть использованы в чат-ботах или других приложениях. 

    \item \textbf{CPI} (метрика) (\textit{Надежда Демина}) "--- 
    стоимость привлечения одного нового пользователя в продукт или сервис. 
\end{enumerate}

\begin{flushleft} \large\textbf{D} \end{flushleft}

\begin{enumerate}
    \setcounter{enumi}{6}

    \item \textbf{Deep Learning} (\textit{Алексей Кузьмин}) "--- 
    подход к машинному обучению, который использует глубокие нейронные сети для решения сложных задач.

    \item \textbf{Data Mining} (\textit{Алексей Кузьмин}) "--- 
    процесс автоматического извлечения ценной информации из больших объемов данных. 

    \item \textbf{Data Science} (\textit{Алексей Кузьмин}) "--- 
    наука о данных, объединяющая разные области знаний: информатику, математику и системный анализ. Сюда входят методы обработки больших данных (Big Data), интеллектуального анализа данных (Data Mining), статистические методы, методы искусственного интеллекта, в т.\,ч. машинное обучение (Machine Learning).

    \item \textbf{DevOps} (акр. от <<development \& operations>>) (\textit{Алексей Кузьмин}) "--- 
    методология автоматизации технологических процессов сборки, настройки и развёртывания программного обеспечения. Методология предполагает активное взаимодействие специалистов по разработке со специалистами по информационно"=технологическому обслуживанию и взаимную интеграцию их технологических процессов друг в друга для обеспечения высокого качества программного продукта.

    \item \textbf{Docker} (\textit{Никита Барабанов}) "--- 
    платформа для контейнеризации приложений, которая позволяет создавать и запускать контейнеры, управлять ими. 
\end{enumerate}

\begin{flushleft} \large\textbf{F} \end{flushleft}

\begin{enumerate}
    \setcounter{enumi}{11}

    \item \textbf{Fullstack-разработчик} (\textit{Павел Пасеков}) "--- 
    специалист, который занимаются разработкой и поддержкой программного обеспечения на всех уровнях стека технологий, включая фронтенд, бэкенд и базы данных.

\end{enumerate}

\begin{flushleft} \large\textbf{G} \end{flushleft}

\begin{enumerate}
    \setcounter{enumi}{12}

    \item \textbf{GAN} (Generative Adversarial Networks) (\textit{Алексей Кузьмин}) "--- 
    тип нейронных сетей, используемый для генерации новых данных, таких как изображения и тексты. 

    \item \textbf{GitLab} (\textit{Никита Барабанов}) "--- 
    платформа для управления кодом и совместной работы разработчиков.
\end{enumerate}

\begin{flushleft} \large\textbf{K} \end{flushleft}

\begin{enumerate}
    \setcounter{enumi}{14}

    \item \textbf{Kubernetes} (\textit{Алексей Кузьмин}) "--- 
    открытое программное обеспечение для автоматизации развёртывания, масштабирования и управления контейнеризированными приложениями (см. также <<Docker>>).
\end{enumerate}

\begin{flushleft} \large\textbf{L} \end{flushleft}

\begin{enumerate}
    \setcounter{enumi}{15}

    \item \textbf{LTV} (метрика) (\textit{Надежда Демина}) "--- 
    показатель, который отражает доход, который компания получит от одного пользователя за всё время его использования продукта или сервиса.
\end{enumerate}

\begin{flushleft} \large\textbf{M} \end{flushleft}

\begin{enumerate}
    \setcounter{enumi}{16}

    \item \textbf{MLOps} (\textit{Алексей Кузьмин}) "--- 
    методология, которая объединяет процессы машинного обучения и операционной деятельности для автоматизации развертывания и управления моделями машинного обучения.
\end{enumerate}

\begin{flushleft} \large\textbf{O} \end{flushleft}

\begin{enumerate}
    \setcounter{enumi}{17}

    \item \textbf{Open Source} (\textit{Алексей Кузьмин}) "--- 
    программное обеспечение с открытым исходным кодом, которое может быть свободно использовано и изменено. 
\end{enumerate}

\begin{flushleft} \large\textbf{R} \end{flushleft}

\begin{enumerate}
    \setcounter{enumi}{18}

    \item \textbf{Ragdoll-физика} (\textit{Леонид Сорокин}) "--- 
    моделирование движения человеческого тела в играх, при котором персонажи отображаются как инертные объекты (словно тряпичные куклы), которые реагируют на удары, гравитацию и другие физические воздействия так же, как это происходит в реальной жизни.  

    \item \textbf{RESTful API} (\textit{Никита Барабанов}) "--- 
    интерфейс программирования приложений, основанный на протоколе HTTP и предоставляющий доступ к ресурсам через стандартные HTTP-методы. 

    \item \textbf{Retention} (метрика) (\textit{Надежда Демина}) "--- 
    показатель, который отражает процент пользователей, которые продолжают использовать продукт или сервис в течение определенного периода времени.
\end{enumerate}

\begin{flushleft} \large\textbf{S} \end{flushleft}

\begin{enumerate}
    \setcounter{enumi}{21}

    \item \textbf{Scala} (\textit{Алексей Кузьмин}) "--- 
    язык программирования, который сочетает в себе функциональное и объектно-ориентированное программирование и часто используется в сфере больших данных.  

    \item \textbf{Scrum} (\textit{Никита Барабанов}) "--- 
    гибкий методология разработки программного обеспечения, основанная на итеративном и инкрементальном подходе к разработке.  

    \item \textbf{SOA (Service Oriented Architecture)} (\textit{Никита Барабанов}) "--- 
    подход к разработке программного обеспечения, основанный на создании сервисов, которые могут быть использованы другими приложениями.

    \item \textbf{Spark} (\textit{Алексей Кузьмин}) "--- 
    фреймворк для обработки больших объемов данных, который позволяет выполнять распределенные вычисления на кластере компьютеров.
\end{enumerate}

\begin{flushleft} \large\textbf{T} \end{flushleft}

\begin{enumerate}
    \setcounter{enumi}{25}

    \item \textbf{Team Lead} (\textit{Павел Пасеков}) "--- 
    руководитель команды разработчиков, который отвечает за организацию работы и координацию усилий команды для достижения общих целей. 

    \item \textbf{TensorFlow} (\textit{Алексей Кузьмин}) "--- 
    библиотека от Google для машинного обучения, которая позволяет создавать и обучать нейронные сети.  
\end{enumerate}

\begin{flushleft} \large\textbf{U} \end{flushleft}

\begin{enumerate}
    \setcounter{enumi}{27}

    \item \textbf{UE (Unreal Engine)} (\textit{Леонид Сорокин}) "--- 
    игровой движок, разработанный компанией Epic Games, который используется для создания видеоигр и визуализации 3D-графики. 

    \item \textbf{UFED} (\textit{Игорь Юрин}) "--- 
    система для сбора данных с мобильных устройств, которые были заблокированы или утеряны.
    
    \item \textbf{UX/UI} (\textit{Алексей Кузьмин}) "--- 
    сокращение от User Experience и User Interface соответственно, обозначающее процесс проектирования и создания интерфейса и опыта пользователя взаимодействия с продуктом.
\end{enumerate}

\begin{flushleft} \large\textbf{V} \end{flushleft}

\begin{enumerate}
    \setcounter{enumi}{30}

    \item \textbf{VPN} (\textit{Алексей Кузьмин}) "--- 
    виртуальная частная сеть, которая позволяет устанавливать безопасное соединение между удаленными компьютерами. 
\end{enumerate}

\begin{flushleft} \large\textbf{W} \end{flushleft}

\begin{enumerate}
    \setcounter{enumi}{31}

    \item \textbf{Waterfall} (\textit{Никита Барабанов}) "--- 
    классическая методология разработки программного обеспечения, основанная на последовательном выполнении этапов разработки. 
\end{enumerate}

\begin{flushleft} \large\textbf{Y} \end{flushleft}

\begin{enumerate}
    \setcounter{enumi}{32}

    \item \textbf{YT} (\textit{Надежда Демина}) "--- 
    фреймворк для обработки и анализа больших объемов данных, используемый в индустрии машинного обучения и аналитики данных.
\end{enumerate}

\begin{flushleft} \large\textbf{А} \end{flushleft}

\begin{enumerate}
    \setcounter{enumi}{33}

    \item \textbf{Актуализация данных} (\textit{Надежда Демина}) "--- 
    процесс обновления информации в базе данных или других хранилищах данных для обеспечения их актуальности и точности.

    \item \textbf{Аппроксимация} (\textit{Алексей Кузьмин}) "--- 
    математический метод описания наблюдаемых закономерностей на основе некоторой функции или модели.

    \item \textbf{Аутсорс компания} (\textit{Павел Пасеков}) "--- 
    компания, которая предоставляет услуги по разработке программного обеспечения или другим ИТ-услугам для других компаний или организаций. 
\end{enumerate}

\begin{flushleft} \large\textbf{Б} \end{flushleft}

\begin{enumerate}
    \setcounter{enumi}{36}

    \item \textbf{Бигдата (Big Data)} (\textit{Надежда Демина}) "--- 
    термин, который описывает большие объемы данных, которые не могут быть эффективно обработаны с помощью традиционных методов и инструментов обработки данных.

    \item \textbf{Билд-сервер} (…непрерывной интеграции) (\textit{Надежда Демина}) "--- 
    сервер, который автоматически собирает и тестирует новые версии программного обеспечения при каждом изменении в исходном коде. 

    \item \textbf{Бэкграунд} (\textit{Алексей Кузьмин}) "--- 
    опыт работы или знания в определенной области, который может помочь в выполнении задач. 
\end{enumerate}

\begin{flushleft} \large\textbf{Г} \end{flushleft}

\begin{enumerate}
    \setcounter{enumi}{39}

    \item \textbf{Гибридный график работы} (\textit{Никита Барабанов}) "--- 
    формат работы, который сочетает в себе удаленную и офисную работу.

    \item \textbf{Графы} (\textit{Галина Громова}) "--- 
    структуры данных, которые представляют собой набор вершин и ребер, используемых для моделирования связей между объектами.

    \item \textbf{Грейды (Grades)} (\textit{Никита Барабанов}) "--- 
    система оценки уровня квалификации и опыта специалистов (джун, мидл, сениор - все сюда). 
\end{enumerate}

\begin{flushleft} \large\textbf{Д} \end{flushleft}

\begin{enumerate}
    \setcounter{enumi}{42}

    \item \textbf{Декомпозиция} (задачи) (\textit{Никита Барабанов}) "--- 
    процесс разбиения сложной задачи на более мелкие и управляемые компоненты.

    \item \textbf{Деплой (Deploy)} (\textit{Никита Барабанов}) "--- 
    процесс установки и настройки новой версии приложения или системы на сервере или в облаке.

    \item \textbf{Джуниор (Junior)} (\textit{Алексей Кузьмин}) "--- 
    начинающий специалист, который только начинает свою карьеру в определенной области.

    \item \textbf{Дискриминация} (\textit{Алексей Кузьмин}) "--- 
    процесс выделения признаков, которые отличают объекты разных классов друг от друга. 
\end{enumerate}

\begin{flushleft} \large\textbf{Ж} \end{flushleft}

\begin{enumerate}
    \setcounter{enumi}{46}

    \item \textbf{Жизненный цикл ПО (Software Development Life Cycle)} (\textit{Никита Барабанов}) "--- 
    последовательность этапов разработки программного обеспечения, начиная от планирования и заканчивая тестированием и поддержкой. 
\end{enumerate}

\begin{flushleft} \large\textbf{З} \end{flushleft}

\begin{enumerate}
    \setcounter{enumi}{47}

    \item \textbf{Запушить} (от англ. <<push>>) (\textit{Никита Барабанов}) "--- 
    процесс отправки изменений в коде в центральный репозиторий.
\end{enumerate}

\begin{flushleft} \large\textbf{И} \end{flushleft}

\begin{enumerate}
    \setcounter{enumi}{48}

    \item \textbf{Интернет вещей (IoT)} (\textit{Игорь Юрин}) "--- 
    сеть устройств, которые могут обмениваться данными и управляться удаленно через Интернет. 

    \item \textbf{Информационная безопасность} (\textit{Игорь Юрин}) "--- 
    комплекс мер и технологий, направленных на защиту информации от утечки, кражи или порчи.
\end{enumerate}

\begin{flushleft} \large\textbf{К} \end{flushleft}

\begin{enumerate}
    \setcounter{enumi}{50}

    \item \textbf{Кибер-кэжуал (Hyper-casual)} (\textit{Надежда Демина}) "--- 
    жанр простых и легко усваиваемых игр для мобильных устройств, которые не требуют сложного геймплея или длительного обучения.

    \item \textbf{Классификация} (\textit{Алексей Кузьмин}) "--- 
    процесс разделения объектов на группы (классы) на основе их характеристик и признаков.

    \item \textbf{Кластеризация} (\textit{Галина Громова}) "--- 
    процесс разделения объектов на группы (кластеры) на основе их сходства между собой.

    \item \textbf{Кодер} (\textit{Алексей Кузьмин}) "--- 
    это человек, который пишет код по четко определенным спецификациям и выбранным заранее алгоритмам, тогда как программист эти алгоритмы создает.

    \item \textbf{Компоненты сильной/слабой связности} (\textit{Галина Громова}) "--- 
    части графа, которые имеют много связей между собой (сильная связность) или мало связей (слабая связность).

    \item \textbf{Компьютерная безопасность} (\textit{Игорь Юрин}) "--- 
    область знаний и практик, связанных с защитой компьютерных систем и данных от несанкционированного доступа, кражи и атак.

    \item \textbf{Конечный автомат} (\textit{Леонид Сорокин}) "--- 
    математическая модель, которая описывает поведение системы или процесса, состоящего из конечного числа состояний и переходов между ними. 

    \item \textbf{Контейнеризация (Containerization)} (\textit{Никита Барабанов}) "--- 
    процесс упаковки приложений и их зависимостей в контейнеры, которые могут быть легко переносимы и масштабируемы. 

    \item \textbf{Криптография} (\textit{Игорь Юрин}) "--- 
    наука о методах защиты информации при помощи шифрования и дешифрования. 
\end{enumerate}

\begin{flushleft} \large\textbf{Л} \end{flushleft}

\begin{enumerate}
    \setcounter{enumi}{59}

    \item \textbf{Лиды (Leads)} (\textit{Алексей Кузьмин}) "--- 
    руководители или менеджеры, которые отвечают за управление проектами или командами в определенной области. 

    \item \textbf{Логи (Logs)} (\textit{Никита Барабанов}) "--- 
    записи в лог-файлах, содержащие информацию о работе приложения или системы. 

    \item \textbf{Логирование (Logging)} (\textit{Никита Барабанов}) "--- 
    процесс записи информации о работе приложения или системы в лог-файлы для последующего анализа.
\end{enumerate}

\begin{flushleft} \large\textbf{М} \end{flushleft}

\begin{enumerate}
    \setcounter{enumi}{62}

    \item \textbf{Майнинг} (\textit{Алексей Кузьмин}) "--- 
    процесс вычислительной работы, при котором используются ресурсы компьютера для генерации новых блоков в блокчейн-сети криптовалюты. Майнинг может быть выполнен как легально, так и нелегально, например, в случае использования вычислительных мощностей без согласия владельца компьютера. 

    \item \textbf{Мердж} (\textit{Надежда Демина}) "--- 
    процесс объединения изменений из разных веток разработки в единую версию программного продукта. 

    \item \textbf{Метрики} (\textit{Павел Пасеков}) "--- 
    показатели, которые используются для измерения эффективности и качества работы системы или процесса (в основном с ними работают аналитики). 

    \item \textbf{Микросервисы (Microservices)} (\textit{Алексей Кузьмин}) "--- 
    архитектурный подход, при котором приложение разбивается на небольшие независимые сервисы, каждый из которых выполняет определенную функцию.
\end{enumerate}

\begin{flushleft} \large\textbf{Н} \end{flushleft}

\begin{enumerate}
    \setcounter{enumi}{66}

    \item \textbf{Нейронные сети} (\textit{Алексей Кузьмин}) "--- 
    алгоритмы машинного обучения, которые имитируют работу человеческого мозга и используются для решения задач классификации, регрессии и прогнозирования.
\end{enumerate}

\begin{flushleft} \large\textbf{П} \end{flushleft}

\begin{enumerate}
    \setcounter{enumi}{67}

    \item \textbf{Пайплайн (Pipeline)} (\textit{Никита Барабанов}) "--- 
    автоматизированный процесс сборки, тестирования и развертывания приложения или системы.

    \item \textbf{Персептрон} (\textit{Алексей Кузьмин}) "--- 
    простейшая форма нейронной сети, которая используется для решения задач классификации и регрессии.

    \item \textbf{Предпродакшн (Pre-production)} (\textit{Никита Барабанов}) "--- 
    окружение, в котором проводится тестирование и подготовка новых версий приложения или системы перед выпуском в продакшн. 

    \item \textbf{Продакшн (Production)} (\textit{Никита Барабанов}) "--- 
    окружение, в котором работает конечный пользователь, используя приложение или систему.

    \item \textbf{Профит (Profit)} (\textit{Никита Барабанов}) "--- 
    прибыль, получаемая от бизнеса или проекта. 
\end{enumerate}

\begin{flushleft} \large\textbf{Р} \end{flushleft}

\begin{enumerate}
    \setcounter{enumi}{72}

    \item \textbf{Развертывание (Deployment)} (\textit{Никита Барабанов}) "--- 
    процесс установки и настройки приложения или системы на сервере или в облаке.

    \item \textbf{Рекуррентные сети} (\textit{Алексей Кузьмин}) "--- 
    тип нейронных сетей, которые используются для анализа последовательностей данных, таких как тексты и временные ряды. 

    \item \textbf{Релиз вышел в прод (Release is live)} (\textit{Никита Барабанов}) "--- 
    это означает, что новая версия приложения или системы доступна для конечных пользователей. 

    \item \textbf{Репозиторий (Repository)} (\textit{Никита Барабанов}) "--- 
    хранилище кода, используемое для управления версиями и совместной работы над проектом.

    \item \textbf{Ретроспективы (Retrospectives)} (\textit{Никита Барабанов}) "--- 
    процесс анализа и обсуждения прошлых итераций разработки для улучшения процесса в будущем (например, в конце спринта).

    \item \textbf{Рефакторинг кода (Code refactoring)} (\textit{Никита Барабанов}) "--- 
    процесс улучшения качества кода без изменения его функциональности.
\end{enumerate}

\begin{flushleft} \large\textbf{С} \end{flushleft}

\begin{enumerate}
    \setcounter{enumi}{78}

    \item \textbf{Сверточные сети} (\textit{Алена Коноплева}) "--- 
    тип нейронных сетей, которые используются для анализа изображений и видео.

    \item \textbf{Сеньор (Senior)} (\textit{Никита Барабанов}) "--- 
    опытный специалист в своей области (в частности -- IT), обладающий высоким уровнем знаний и навыков. 

    \item \textbf{Синаптические веса} (или просто веса) (\textit{Алексей Кузьмин}) "--- 
    числовые значения, которые определяют важность связей между нейронами в нейронной сети. 

    \item \textbf{Спринт} (\textit{Никита Барабанов}) "--- 
    короткий временной интервал, в течение которого scrum-команда выполняет заданный объем работы (обычно 2 недели). 

    \item \textbf{Сэмпл} (\textit{Алексей Кузьмин}) "--- 
    часть набора данных, используемая для обучения модели машинного обучения. 
\end{enumerate}

\begin{flushleft} \large\textbf{Т} \end{flushleft}

\begin{enumerate}
    \setcounter{enumi}{83}

    \item \textbf{Тимбилдинг (Team Building)} (\textit{Алексей Кузьмин}) "--- 
    процесс формирования и укрепления команды для достижения общих целей.

    \item \textbf{Троянцы} (\textit{Игорь Юрин}) "--- 
    вредоносные программы, которые могут скрыться в компьютерной системе и выполнять различные действия без ведома пользователя. 
\end{enumerate}

\begin{flushleft} \large\textbf{У} \end{flushleft}

\begin{enumerate}
    \setcounter{enumi}{85}

    \item \textbf{Унификация сигналов} (\textit{Надежда Демина}) "--- 
    процесс приведения различных типов сигналов к единому формату для обеспечения их совместимости и обработки. 
\end{enumerate}

\begin{flushleft} \large\textbf{Ф} \end{flushleft}

\begin{enumerate}
    \setcounter{enumi}{86}

    \item \textbf{Фастдата (Fast Data)} (\textit{Алексей Кузьмин}) "--- 
    технология обработки потоков данных в режиме реального времени, которая позволяет быстро и эффективно анализировать большие объемы данных, поступающих в систему.

    \item \textbf{Фреймворки} (\textit{Алексей Кузьмин}) "--- 
    набор инструментов и библиотек, которые используются для разработки программного обеспечения.
\end{enumerate}

\begin{flushleft} \large\textbf{Ц} \end{flushleft}

\begin{enumerate}
    \setcounter{enumi}{88}

    \item \textbf{Цикл разработки (Development Cycle)} (\textit{Алексей Кузьмин}) "--- 
    последовательность этапов, которые проходит проект в процессе его разработки, от идеи до выпуска готового продукта. 
\end{enumerate}

\begin{flushleft} \large\textbf{Э} \end{flushleft}

\begin{enumerate}
    \setcounter{enumi}{89}

    \item \textbf{Эпоха обучения} (\textit{Алексей Кузьмин}) "--- 
    один цикл обучения нейронной сети на наборе данных. 
\end{enumerate}

\begin{flushleft} \large\textbf{Ю} \end{flushleft}

\begin{enumerate}
    \setcounter{enumi}{90}

    \item \textbf{Юнит-тесты (Unit tests)} (\textit{Никита Барабанов}) "--- 
    тесты, которые проверяют отдельные компоненты приложения или системы на соответствие требованиям. 
\end{enumerate}