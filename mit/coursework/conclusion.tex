В ходе данной работы:

\begin{enumerate}
    \item Были изучены теоретические основы построения лексических и
    синтаксических анализаторов.
    \item Проанализированы особенности реализации лексических и синтаксических
    анализаторов.
    \item Были изучены принципы работы генераторов лексического и
    синтаксического анализа на примере Flex и GNU Bison.
    \item Были созданы лексический и синтаксический анализаторы для анализа
    математического выражения.
    \item Было изучено понятие абстрактного синтаксического дерева.
    \item Проведен анализ производительности полученных реализаций.
\end{enumerate}

Таким образом, все поставленные в рамках работы задачи выполнены.

Результаты исследования показали, что абстрактные синтаксические деревья
позволяют добиться увеличения производительности в $5$--$6$ раз.

А это, в свою очередь, позволяет утверждать о том, что концепция абстрактных
синтаксических деревьев является крайне важной в информатике и ее приложениях, в
частности, при создании синтаксических анализаторов.