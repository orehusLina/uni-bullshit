Формально определим требования к системе:
\begin{enumerate}
    \item Регион должен представлять из себя некоторый непрерывный участок
    размера $n$ байт (в начальный момент времени размер равен некоторой
    начальной величине $n_0$).
    \item При обращении к региону он должен предоставить $k$ байт памяти и
    вернуть некоторый идентификатор этого участка для последующего обращения.
    \item При заполнении региона должна быть возможность увеличить объем
    доступной памяти в некоторое число раз, которое далее будем называть
    коэффициентом увеличения.
    \item Должна быть доступна возможность эффективного освобождения всей
    выделенной регионом памяти.
\end{enumerate}

Единственной сложной операцией над регионом является его увеличение. Так как
выделение нового участка потенциально может сопровождаться изменением адресов
объектов, то необходимо организовать доступ к ним независимо от первоначального
адреса. Для этого для каждого объекта будем получать доступ к нему через
некоторый индекс.

Кроме того, коэффициент увеличения должен быть выбран таким образом, чтобы был
соблюден баланс между оптимальным объемом выделенной памяти и частотой системных
вызовов.