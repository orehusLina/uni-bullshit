\documentclass[bachelor, och, otchet]{SCWorks}
\usepackage[T2A]{fontenc}
\usepackage[utf8]{inputenc}
\usepackage{graphicx}
\usepackage[sort,compress]{cite}
\usepackage{amsmath}
\usepackage{amssymb}
\usepackage{amsthm}
\usepackage{fancyvrb}
\usepackage{longtable}
\usepackage{array}
\usepackage[english,russian]{babel}
\usepackage{minted}
\usepackage{tempora}
\usepackage[colorlinks=false]{hyperref}

\setminted[cpp]{fontsize=\small, breaklines=true, style=bw, linenos}
\setminted[python]{fontsize=\small, breaklines=true, style=bw, linenos}

\newcommand{\eqdef}{\stackrel {\rm def}{=}}

\newtheorem{lem}{Лемма}

% % При использовании biblatex вместо bibtex
%\usepackage[style=gost-numeric]{biblatex} \addbibresource{thesis.bib}

\begin{document}

% Кафедра (в родительном падеже)
\chair{математической кибернетики и компьютерных наук}

% Тема работы
\title{ЛЕКСИЧЕСКИЙ И СИНТАКСИЧЕСКИЙ АНАЛИЗ ВЫРАЖЕНИЙ}

% Курс
\course{2}

% Группа
\group{251}

% Факультет (в родительном падеже) (по умолчанию "факультета КНиИТ")
%\department{факультета КНиИТ}

% Специальность/направление код - наименование \napravlenie{02.03.02 "---
%Фундаментальная информатика и информационные технологии} \napravlenie{02.03.01
%"--- Математическое обеспечение и администрирование информационных систем}
%\napravlenie{09.03.01 "--- Информатика и вычислительная техника}
\napravlenie{09.03.04 "--- Программная инженерия}
%\napravlenie{10.05.01 "--- Компьютерная безопасность}

% Для студентки. Для работы студента следующая команда не нужна.
%\studenttitle{Студентки}

% Фамилия, имя, отчество в родительном падеже
\author{Рыданова Никиты Сергеевича}

% Заведующий кафедрой
\chtitle{доцент, к.\,ф.-м.\,н.} % степень, звание
\chname{С.\,В.\,Миронов}

%Научный руководитель (для реферата преподаватель проверяющий работу)
\satitle{доцент, к.\,ф.-м.\,н.} %должность, степень, звание
\saname{Г.\,Г.\,Наркайтис}

% Руководитель практики от организации (только для практики, для остальных типов
% работ не используется)
\patitle{к.\,ф.-м.\,н., доцент}
\paname{Д.\,Ю.\,Петров}

% Семестр (только для практики, для остальных типов работ не используется)
\term{2}

% Наименование практики (только для практики, для остальных типов работ не
% используется)
\practtype{учебная}

% Продолжительность практики (количество недель) (только для практики, для
% остальных типов работ не используется)
\duration{2}

% Даты начала и окончания практики (только для практики, для остальных типов
% работ не используется)
\practStart{01.07.2016} \practFinish{14.07.2016}

% Год выполнения отчета
\date{2021}

%\maketitle

% Включение нумерации рисунков, формул и таблиц по разделам (по умолчанию -
% нумерация сквозная) (допускается оба вида нумерации) \secNumbering


%\tableofcontents

% Раздел "Обозначения и сокращения". Может отсутствовать в работе \abbreviations
% \begin{description} \item ... "--- ... \item ... "--- ... \end{description}

% Раздел "Определения". Может отсутствовать в работе \definitions

% Раздел "Определения, обозначения и сокращения". Может отсутствовать в работе.
% Если присутствует, то заменяет собой разделы "Обозначения и сокращения" и
% "Определения" \defabbr


% Раздел "Введение"

%\intro

% После введения — серии \section, \subsection и т.д.
\section{Абстрактные синтаксические деревья}
\subsection{Управление памятью на основе регионов}

\subsubsection{Мотивировка}
\input{motivirovka.tex}

\subsubsection{Построение}
\input{postroenie.tex}

\subsubsection{Определение структуры}
\input{def_structute.tex}

\subsubsection{Инициализация}
\input{initialization.tex}

\subsubsection{Выделение памяти}
\input{memory_allocation.tex}

\subsubsection{Освобождение выделенной памяти}
Наконец, реализуем освобождение выделенной региону памяти с помощью функции
\verb|arena_free|

\begin{minted}{cpp}
void arena_free (arena* arena) { 
    if (arena->arena != NULL) 
        free(arena->arena);
    arena->arena = NULL; 
}
\end{minted}

\subsubsection{Модификация абстрактного синтаксического дерева}
Осталось изменить исходный код программы, чтобы обеспечить выделение памяти с
помощью полученной нами структуры данных.

Для этого воспользуемся директивой \verb|%param| и заявим в качестве параметра
переменную типа \verb|arena*|. В функциях \verb|eval|, \verb|newnum|,
\verb|newast| внесем изменения, чтобы обеспечить выделение памятью с помощью
написанных ранее функций.

С полным кодом программы можно ознакомиться в приложении А.

\subsubsection{Сборка проекта}
Теперь проект можно собрать, незначительно изменив \verb|Makefile|:

\begin{minted}[fontsize=\small, breaklines=true, style=bw, linenos]{shell}
calc.out: calc.l calc.y arena_ast.h 
    bison -d calc.y 
    flex calc.l 
    cc -o $@ calc.tab.c lex.yy.c arena_ast.c arena.c
\end{minted}
и запустить. Результат работы программы представлен на рис. 1

\begin{figure}[hbt!]
    \centering
    \includegraphics[scale=0.5]{naivetest.png}
    \caption{Демонстрация работы программы}
    \label{fig:naivetest}
\end{figure}

\section{Сравнение полученных реализаций}
Проведем анализ производительности полученных версий анализатора. В качестве
данных для тестирования возьмем выражения вида $\underbrace{2 + 2 + 2 \dots +
2}_{n}$ для $n = 1\dots100$ с шагом $1$. Для вычисления времени выполнения
воспользуемся библиотекой \verb|time| Python 3.9.5. Автоматизацию обеспечим с
помощью библиотеки \verb|subprocess|. Получим следующий код:

\inputminted{python}{test.py}

Кроме того, отметим, что в ранее написанные программы были внесены некоторые
изменения для проведения эксперимента. Ознакомиться с ними можно в приложении А.

Ознакомиться с полным исходным кодом программы, осуществляющей исследование
производительности можно в приложении Б.

Для большей наглядности графики интерполированы полиномом с помощью функции
\verb|polyfit| библиотеки \verb|numpy|.

Ознакомиться с полным исходным кодом программы, осуществляющей анализ полученных
результатов можно в приложении В.

Результаты исследования изображены на рис. 2:

\begin{figure}[hbt!]
    \centering
    \includegraphics[scale=0.5]{benchmark.png}
    \caption{Сравнение полученных результатов}
    \label{fig:benchmark}
\end{figure}

Исследование показало, что использование абстрактных синтаксических деревьев
позволяет уменьшить время работы программы более чем в $5$ раз, что существенно
заметно для выражений любой длины.

Также из графиков видно, что в рамках данной работы не удалось добиться большей
производительности при управлении памятью на основе регионов. Тем не менее, она
все еще может считаться более предпочительной ввиду перечисленных ранее
преимуществ.

% Раздел "Заключение"
\conclusion
\input{conclusion.tex}

%Библиографический список, составленный вручную, без использования BibTeX
%
%\begin{thebibliography}{99} \bibitem{Ione} Источник 1. \bibitem{Itwo} Источник
%  2 \end{thebibliography}

%Библиографический список, составленный с помощью BibTeX
%

\inputencoding{cp1251}
\bibliographystyle{gost780uv}
\bibliography{thesis}
\inputencoding{utf8}

% % При использовании biblatex вместо bibtex \printbibliography

% Окончание основного документа и начало приложений Каждая последующая секция
% документа будет являться приложением
\appendix 
\section{Flash-носитель с исходным кодом программ, использующихся в работе}
\noindent \textbf{Папка} \verb|src| содержит оригинальный исходный код
программы:

\textbf{Папка} \verb|naive| — реализация без АСД

\textbf{Папка} \verb|naiveast| — реализация с АСД

\textbf{Папка} \verb|arena| — реализация с АСД на основе региона

\noindent \textbf{Папка} \verb|extsrc| содержит измененный исходный код,
необходимый для исследования производительности:

\textbf{Папка} \verb|naive| — реализация без АСД

\textbf{Папка} \verb|naiveast| — реализация с АСД

\textbf{Папка} \verb|arena| — реализация с АСД на основе региона


\section{Исходный код программы на Python, осуществляющей исследование 
производительности полученных реализаций}

\inputminted{python}{test.py}

\section{Исходный код программы на Python, осуществляющей анализ полученных
результатов}

\inputminted{python}{graph.py}

\end{document}