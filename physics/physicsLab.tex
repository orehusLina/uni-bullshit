\documentclass[12pt]{article}

\usepackage[utf8]{inputenc}
\usepackage[T2A]{fontenc}
\usepackage[russian]{babel}

\usepackage{amsmath}
\usepackage{ amssymb }
\usepackage{amsfonts}
\usepackage{ gensymb }
\usepackage{amsthm}
\usepackage{icomma}
\usepackage{float}

\usepackage{multirow}
\usepackage{graphicx}
\usepackage{wrapfig}

\usepackage[a4paper, left=2cm, right=2cm, bottom=2.5cm]{geometry}

\usepackage[shortlabels]{enumitem}
\usepackage{fancyhdr}

\newcommand\field[1]{\noindent\textbf{#1}\ }

\pagestyle{fancy}

\lhead{Стюхина А., Орехова А.}
\rhead{Лабораторная работа №2}

\setlist[enumerate]{itemsep=0mm}

\author{Стюхина Ангелина, Орехова Алина}
\date{22 марта 2023}
\title{Лабораторная работа №2}

\begin{document}
\maketitle

\field{Наименование работы:} определение скорости звука в воздухе методом интерференции.

\field{Цель работы:} изучение процесса распространения звуковой волны в газе и измерение скорости звука при различных случаях интерференции волн.

\field{Принадлежности:} прибор Квинке, звуковой генератор.

\section*{Краткая теория}
\begin{wrapfigure}{r}{0.5\textwidth}
    \centering\includegraphics[width=0.45\textwidth]{chertej.png}
    \caption{Чертеж установки. Прибор Квинке}
    \label{fig:oberbeque}
\end{wrapfigure}

На явлении интерференции основано определение скорости звука в воздухе с помощью прибора Квинке.

Основной частью этого прибора (рис. 1) являются две изогнутые латунные трубки 1 и 2, укрепленные вертикально параллельно друг другу на стойке 3. Длина трубки 2 может изменяться выдвижением ее. Удлинение трубки 2 определяется с помощью указателя 4 по шкале, нанесенной на стойке. Входные концы трубок 1 и 2 подсоединены к тройнику 9, другой конец которого закрыт мембраной телефонной трубки 8. Выводы от телефонной трубки подсоединяются к выходу генератора звуковой частоты 5. Выходные концы трубок 1 и 2 подсоединены к тройнику 6, на другой конец которого надета резиновая трубка с эбонитовым наконечником 7.

Звуковая волна, возбуждаемая мембраной телефона 8, колеблющейся с частотой, задаваемой звуковым генератором, поступает на вход тройника 9 и разветвляется на его выходе на две части. Таким образов добиваются когерентности волн. Одна волна проходит по трубке 1, другая по трубке 2. Если длину трубки 2 увеличить по сравдению с длиной трубки 1, то волны, соединяясь вместе на выходе из тройника 6, будут иметь разность хода, так как пути, пройденные ими, будут не равны. В зависимости от величины разности хода в слуховой трубке 7 будут слышны усиления или ослабления звука.

Пусть при первом минимуме звука положение указателя 4 на шкале равно $a_1$. Тогда в соответствии с конструкцией установки разность хода волн будет равна $2a_1$. В этом случае выполняется соотношение $2a_1 = \frac{\lambda}{2}$, отсюда

\begin{equation}
    \lambda = 4a_1\cdot
\end{equation}

Так как скорость звука $\upsilon$ связана с длиной волны $\lambda$  и частотой $\mathcal{V}$ соотношением $\upsilon = \lambda \mathcal{V}$, то имеем 

\begin{equation}
    \upsilon = 4a_1 \mathcal{V} \cdot
\end{equation}

Если $a_2$ соответствует положению указателя при втором минимуме, то разность хода волн запишется равенством $2a_2 = 3 \frac{\lambda}{2}$, и, следовательно, скорость звука определится из соотношения 

\begin{equation}
    \upsilon = \frac{4}{3} a_2 \mathcal{V} \cdot
\end{equation}

Для третьего минимума звука соответственно будем иметь $2a_3 = 5 \frac{\lambda}{2}$ и

\begin{equation}
    \upsilon = \frac{4}{5} a_3 \mathcal{V} \cdot
\end{equation}

Остальные минимумы вычисляются аналогичным путем.

В то же время, скорость распространения продольных звуковых волн в упругой среде определяется соотношением 
\begin{equation}
    v = \sqrt{\frac{\gamma p}{\rho}},
\end{equation}
где $\rho$ - плотность среды, p - давление, $\gamma$ - некоторая константа, для воздуха имеющая значение 1,41. Это соотношение называется формулой Лапласа.

Скорость звука, посчитанная по формуле Лапласа, в воздухе ($\gamma = 1.41$) для плотности $\rho_0 = 1.293 \cdot 10^{-3}$ $\text{г/см}^3$ (при $t = 0 \degree C$) и атмосферного давления $p_0 = 1.013 \cdot 10^5$ Па, равна 332 м/с, что согласуется с полученным экспериментальным путем значением 331.46 м/с. 

Так как при данной температуре  давление р и плотность $\rho$ изменяются пропорционально друг другу, то, как показывает формула Лапласа, скорость звука не зависит от давления в газе. Из этой же формулы следует, что скорость звука в газах существенно зависит от температуры среды. Действительно, плотность среды зависит от температуры по закону $\rho = \frac{\rho_0}{1+\alpha t}$, где $\rho$ - плотность при температуре $t \degree C$, $\rho_0$ - плотность при $0 \degree C$, $\alpha$ - коэффициент расширения газа, равный 0.004 $\text{град}^{-1}$. Отсюда
\begin{equation}
    v = \sqrt\frac{\gamma p_0}{\rho_0}\cdot\sqrt{1+\alpha t}, 
\end{equation} 
или
\begin{equation}
    v_0 = \frac{v}{\sqrt{1+\alpha t}},
\end{equation}
где $\rho_0$ - нормальное атмосферное давление, $v$ - скорость звука при температуре $t \degree C$, $v_0$ - скорость звука при $0 \degree C$.

\section*{Экспериментальная часть}
\begin{enumerate}
    \item Включить в сеть звуковой генератор.
    \item Установить одну из трех заданных преподавателем частот от 1400 до 2000 Гц.
    \item Установить одинаковую длину трубок 1 и 2.
    \item Выдвигая трубку 2 и наблюдая изменения громкость звука с помощью слуховой трубки 7 на выходе тройника 6, заметить положение $a_1$ указателя 4 на шкале при 1-м минимуме звука.
    \item Проделав операции, указанные в п. 4, определить значения $a_2$, $a_3$ и т.д. для последующих минимумов.
    \item Провести измерения, указанные в п.п. 4-5 для остальных частот.
    \item Вычилить значения скорости звука $\overline{v}$ по формулам вида (2)-(4).
    \item Вычилить среднее значение скорости звука $\overline{v}$ по данным всех измерений при данной температуре.
    \item Рассчитать по формуле (7) скорость звука в воздухе при $0 \degree C$.
    \item Сравнить полученное значение скорости звука при $0 \degree C$ с табличным значением, равным $\upsilon_0 = 331,46$ м/с при  $0 \degree C$ и давлении воздуха 1013,25 гПа, и дать объяснение возможному расхождению значений.
    \item Данные прямых измерений и вычислений занести в таблицу.
\end{enumerate}

\field{Результаты опыта:}

\begin{center}
    \begin{tabular}{|l|l|l|l|l|l|l|l|c|l|c|}
        \hline
        $\mathcal{V}$, Гц & $a_1$, м & $v_1$,м/с & $a_2$, м & $v_2$,м/с & $a_3$, м & $v_3$,м/с & $\overline{v}$, м/с & $\overline{v}$, м/с  & $v_0$, м/с & \multicolumn{1}{l|}{$\overline{v_0}$, м/с} \\ \hline
        800             & 0.108  & 345.6     & 0.322  & 343.5    & -      & -        & 344.53            & \multirow{13}{*}{348.09} & 327.31          & \multirow{13}{*}{330.69}                        \\ \cline{1-8} \cline{10-10}
        900             & 0.096  & 345.6     & 0.287  & 344.4    & -      & -        & 345.00            &                          & 327.75          &                                                 \\ \cline{1-8} \cline{10-10}
        1000            & 0.087  & 348.0     & 0.259  & 345.3    & -      & -        & 346.67            &                          & 329.34          &                                                 \\ \cline{1-8} \cline{10-10}
        1100            & 0.078  & 343.2     & 0.235  & 344.7    & 0.390  & 343.2    & 343.69            &                          & 326.51          &                                                 \\ \cline{1-8} \cline{10-10}
        1200            & 0.071  & 340.8     & 0.224  & 358.4    & 0.380  & 364.8    & 354.67            &                          & 336.94          &                                                 \\ \cline{1-8} \cline{10-10}
        1300            & 0.066  & 343.2     & 0.200  & 346.7    & 0.332  & 345.3    & 345.05            &                          & 327.80          &                                                 \\ \cline{1-8} \cline{10-10}
        1400            & 0.062  & 347.2     & 0.186  & 347.2    & 0.314  & 351.7    & 348.69            &                          & 331.26          &                                                 \\ \cline{1-8} \cline{10-10}
        1500            & 0.058  & 348.0     & 0.173  & 346.0    & 0.295  & 354.0    & 349.33            &                          & 331.87          &                                                 \\ \cline{1-8} \cline{10-10}
        1600            & 0.054  & 345.6     & 0.164  & 349.9    & 0.272  & 348.2    & 347.88            &                          & 330.49          &                                                 \\ \cline{1-8} \cline{10-10}
        1700            & 0.051  & 346.8     & 0.153  & 346.8    & 0.254  & 345.4    & 346.35            &                          & 329.03          &                                                 \\ \cline{1-8} \cline{10-10}
        1800            & 0.049  & 352.8     & 0.146  & 350.4    & 0.243  & 349.9    & 351.04            &                          & 333.49          &                                                 \\ \cline{1-8} \cline{10-10}
        1900            & 0.046  & 349.6     & 0.138  & 349.6    & 0.231  & 351.1    & 350.11            &                          & 332.61          &                                                 \\ \cline{1-8} \cline{10-10}
        2000            & 0.044  & 352.0     & 0.131  & 349.3    & 0.222  & 355.2    & 352.18            &                          & 334.57          &                                                 \\ \hline
        \end{tabular}
\end{center}

При измерении всех минимумов $a_1, a_2, a_3$ температура помещения составляла $t = 27 \degree C$. Соответственно, были получены средние скорости звука при данной температуре. Итоговая средняя скорость при $t = 27 \degree C$ составила 348.09 м/с. \par

По формуле $\upsilon_0 = \frac{\upsilon
}{\sqrt{1 + \alpha t}}$, где $\alpha$ - коэффициент расширения газа, равный 0.004 $\text{град}^{-1}$, вычислим для каждой частоты значение скорости звука при нулевой температуре. \newline
\newline \newline
Итоговая средняя скорость звука при нулевой температуре составит 

\begin{equation}
    \overline{\upsilon_0} = \frac{348.09}{\sqrt{1 +  0.004\cdot27}} = 330.69  \text{(м/с)}
\end{equation}
\newline
\field{Анализ полученных результатов:}

Сравним полученное экспериментальным путем значение скорости звука при $t = 0 \degree C$ c табличным значением, равным $\upsilon_0 = 331,46$, т.е. вычислим погрешность:
\begin{equation}
    \delta\overline{\upsilon_0} = \frac{|331.46-330.69|}{331.46} \cdot 100\% = 2 \%
\end{equation}

Погрешность довольно незначительна и вызвана трудностью локализации звукового минимума для человеческого слуха. 
\section*{Дополнительное задание} 
В качестве дополнительного задания предлагалось определить область частот, в которой измерение скорости звука оказалось наиболее точным. \par
Определим наиболее точным измерением такое измерение, в котором относительная погрешность с теоретическим значением скорости звука не превышает 0.5\%. \par
Тогда нам подходят частоты 1400, 1500 и 1600 Гц:
\begin{equation}
    \delta v_{1400} = \frac{|331.46-331.26|}{331.46} \cdot 100\% = 0.06\% 
\end{equation}
\begin{equation}
    \delta v_{1500} = \frac{|331.46-331.87|}{331.46} \cdot 100\% = 0.12\% 
\end{equation}
\begin{equation}
    \delta v_{1600} = \frac{|331.46-330.49|}{331.46} \cdot 100\% = 0.29\% 
\end{equation}

\field{Вывод: }

Метод интерференции позволяет измерить скорость звука в воздухе с погрешностью 2\% и дает наиболее точный результат для частот 1400-1600 Гц.
\end{document}
